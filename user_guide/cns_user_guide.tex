\documentclass[12pt]{report}

%math packages
\usepackage{amssymb}
\usepackage{amsmath}
\usepackage{amstext}
\usepackage{dsfont}

%other packages
\usepackage{natbib}
\usepackage{comment}
\usepackage{graphicx}
\usepackage{float}
\usepackage{epstopdf}
\usepackage[pdftex,colorlinks]{hyperref}
  \hypersetup{linkcolor=black,urlcolor=black,citecolor=black}


%page setup
\usepackage{geometry}
\geometry{letterpaper}
\usepackage{fullpage}

\title{User Guide\\ \vspace{0.1in} \large{Curvilinear Navier-Stokes (CNS) C++ Code}}
\author{Bobby Arthur \\ Department of Civil and Environmental Engineering
                     \\ Environmental Fluid Mechanics Laboratory
                     \\ Stanford University}
\date{\today}

\begin{document}
\maketitle
\tableofcontents

\chapter{Code Structure}

\section{Introduction}
This curvilinear Navier-Stokes (CNS) C++ code is also known as ``cns-koltakov" or, more affectionately, ``Sergey's Code," after its author, Sergey Koltakov. It is the third version of the original Fortran code written by Yan Zang at Stanford in the early 1990's \citep{zang1994}. The original code was serial (one processor). The second version was updated by Cui to run with multiple processors using MPI, and is known as PCUI \citep{cui2001}. This latest version of the code is basically a translation of PCUI to C++ using an object-oriented framework, which allowed for the addition of a moving grid \citep{koltakov2012}. Over the years, these various versions of the code have been used to study a wide range of geophysical flows.

The code is based on the fractional step method of \citet{zang1994} and includes the LES model of \citet{zang1993}\footnote{The older versions of the code include the LES model. At present, it has not been written into the C++ version. Someone should do this.}. It solves the incompressible Navier-Stokes equations on a generalized curvilinear grid with a rigid lid, employing a semi-implicit time integration scheme with Adams-Bashforth for the explicit terms and Crank-Nicholson for the implicit terms. Additionally, it uses the QUICK and SHARP schemes for advection of momentum and scalars, respectively, and solves the pressure Poisson equation with a multigrid method. 

\section{File Summary}
The code is contained in the directory \texttt{/cns-koltakov}, which contains the following files:
\begin{description}
\item[\texttt{parameters.dat}] Hello
\item[\texttt{parameters.h}] Hello
\item[\texttt{navier\_stokes\_solver.h}/\texttt{.cpp}] Hello
\item[\texttt{curvilinear\_grid.h}/\texttt{.cpp}] Hello
\end{description}

\section{Getting Started}
 
\chapter{Code Output}

\section{Binary Output}

\section{Loading and Viewing Output Data}

\chapter{Examples}

\section{Internal Seiche}

\section{Lid-driven Cavity}

\section{Lock Exchange}

\section{Breaking Internal Wave}

\section{Sediment Transport}
Coming soon!

\bibliographystyle{apalike}
\bibliography{cns_user_guide}

\end{document}
